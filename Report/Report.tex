% !TeX spellcheck = <none>
\documentclass[11pt,a4paper]{article}
\usepackage{etex}
\reserveinserts{28}

\usepackage[utf8]{inputenc}

\usepackage{hyperref}
\usepackage{amsthm}
\usepackage{amsmath}
\usepackage{amssymb}
\usepackage{amsfonts}
\usepackage[portuguese]{babel}
\usepackage{pictex}
\usepackage[dvips]{graphics}
\usepackage{enumerate}
\usepackage{color}
\usepackage[usenames,dvipsnames]{xcolor}

\usepackage{amsmath}
\usepackage{amssymb}
\usepackage{amsfonts}
\usepackage[portuguese]{babel}
\usepackage{pictex}
\usepackage[dvips]{graphics}
\usepackage{enumerate}
\usepackage{color}
\usepackage{indentfirst}
\usepackage{listings}
\usepackage{tabularx}
\usepackage{multirow}
\usepackage{float}
\usepackage{makeidx}

\usepackage{graphicx}
\setlength{\parindent}{1cm}
\title{\bf{Compiladores - Projecto}\vspace{50mm}\\iJava\vspace{80mm}}
\author{
João Ricardo Lourenço, Nº 2011151194\\
Joaquim Pedro Bento Gonçalves Pratas Leitão, Nº 2011150072}
\makeindex
\begin{document}

\maketitle
\centerline{\textbf{Relatório}}
\pagebreak

\printindex

\renewcommand*\contentsname{Índice}
\tableofcontents

\pagebreak

\section{Introdução}

O presente trabalho pretende-se desenvolver um compilador para a linguagem $iJava$, um pequeno subconjunto da linguagem Java (versão 5.0). Por ser um subconjunto de uma outra linguagem, todos os programas que respeitem as regras impostas em $iJava$ são também, garantidamente, programas válidos em $Java$.

Nesta linguagem todos os programas são constituídos por uma única classe, que possui métodos e atributos estáticos, e públicos. Para além disso, a classe necessita obrigatoriamente de ter um método $main$, onde a execução do programa se inicia. 

Podemos utilizar literais dos tipos inteiro e booleano e variáveis inteiras, boleanas e arrays unidimensionais de inteiros e booleanos.

A linguagem implementa também expressões aritméticas e lógicas, operações relacionais simples, instruções de atribuição e controlo ($if-else$ e $while$).

Os métodos definidos, e os respectivos valores de retorno, podem receber como argumentos variáveis de qualquer tipo acima mencionado, com excepção do método $main$, que tal como em $Java$ possui como tipo de retorno o tipo $void$.

É também possível passar parâmetros (literais inteiros) ao nosso programa através da linha de comandos. É o método $main$ que vai receber esses parâmetros, armazenando-os num array de objectos do tipo $String$. Embora este tipo de dados não esteja incluído na lista de tipos permitidos em $iJava$, a sua utilização apenas é permitida no método $main$, com a mera finalidade de obter os parâmetros passados ao programa aquando da sua invocação.

O desenvolvimento do compilador foi dividido em três fases distintas.

Numa primeira fase foi realizada a \emph{Análise Lexical} do programa fonte, onde são identificados $tokens$, isto é, cadeias pertencentes à linguagem e que têm significado e relevância para o programa.

Seguiu-se a \emph{Análise Semântica}, composta por quatro etapas principais: FIXME, MUDAR A MANEIRA COMO ESTAMOS A DIZER O QUE SE SEGUE

\begin{itemize}
\item \textbf{Tradução da gramática-fonte} (fornecida em notação $EBNF$) para o yacc, permitindo assim reconhecer se o programa fornecido é constituído por sequências de $tokens$ pertencentes à linguagem, e assim detectar eventuais erros de sintaxe.

\item \textbf{Construção da árvore de sintaxe abstracta}, etapa que é realizada em simultâneo com a verificação e validação do programa a compilar, de acordo com a gramática da nossa linguagem. A árvore de sintaxe abstracta irá representar o nosso programa a compilar, recorrendo a uma estrutura em árvore para representar as estruturas sintáticas das cadeias que constituem o programa a compilar.

\item \textbf{Construção da tabela de símbolos}, utilizadas para armazenar informações relevantes sobre a classe (seus atributos e métodos), bem como sobre cada método definido pelo programador (como, por exemplo, o tipo de retorno e os argumentos).

\item \textbf{Verificação de erros semânticos}, etapa principal da \emph{Análise Semântica}, onde são realizadas verificações de tipos, garantindo que para cada operação a realizar não existem incompatiblidades de tipos entre os operandos nela envolvidos.
\end{itemize}

A última fase do trabalho consistiu na \emph{Geração de Código Intermédio}, da qual resulta, na representação intermédia de $LLVM$, um programa equivalente ao que pretendemos compilar.

METER IMAGEM BONITINHA, TIPO CACEIRO???

\pagebreak

\section{Análise Lexical}

Tal como referimos anteriormente, na \emph{Análise Lexical} procedemos à identificação dos $tokens$ da nossa linguagem. Sempre que é detectada a presença de um comentário no programa a compilar, seja do tipo $// ...$ (comentários de apenas uma linha) ou do tipo $/* ... */$ (comentários multi-linha), os caracteres incluídos nesse comentário são ignorados.

Sempre que é detectado um caracter, ou uma sequência de caracteres, que não constitui nenhum $token$ é detectado um erro lexical, sendo impressa uma mensagem de erro, indicando a existência de um caracter ilegal, juntamente com a sua posição no programa.

Adicionalmente, caso se verifique a ocorrência de um comentário multi-linha que não foi devidamente terminado, o erro lexical é também detectado, sendo impressa uma mensagem de erro que indica a posição no programa onde o comentário foi iniciado.

Em seguida, apresentamos a lista dos $tokens$ válidos na linguagem $iJava$ e a lista dos $tokens$ reservados que, por essa razão, não estão disponíveis na nossa linguagem:

\begin{enumerate}
\item \textbf{BOOLLIT}

\item \textbf{INT}

\item \textbf{BOOL}

\item \textbf{NEW}

\item \textbf{IF}

\item \textbf{ELSE}

\item \textbf{WHILE}

\item \textbf{PRINT}

\item \textbf{PARSEINT}

\item \textbf{CLASS}

\item \textbf{PUBLIC}

\item \textbf{STATIC}

\item \textbf{VOID}

\item \textbf{STRING}
\end{enumerate}


\pagebreak

\section{Análise Sintática}

\pagebreak

\section{Construção da Árvore de Sintaxe Abstracta}

\pagebreak

\section{Análise Semântica}

	\subsection{Tabelas de Símbolos}
	
\pagebreak
	
	\subsection{Detecção de Erros Semânticos}
	
\pagebreak
	
\section{Geração de Código}

\pagebreak

\section{Apreciação do Trabalho}

\end{document}